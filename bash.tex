\documentclass[a4paper,12pts]{article}

\usepackage[utf8]{inputenc}
\setlength\parindent{0pt}

\begin{document}
\title{Bash Cheat Sheet}
\author{Stephane Espitia}
\date{\today}
\maketitle

\pagenumbering{roman}
\tableofcontents
\newpage
\pagenumbering{arabic}


\section{Navigation}
\begin{itemize}
    \item \textit{pwd}: print working directory
    \item \textit{cd}: change directory according to path given
\end{itemize}

\section{Listing Files}
\begin{itemize}
    \item ls: listing a directory content, common option: 
       \begin{itemize}
           \item -l list format
           \item -a all files even hidden ones
       \end{itemize}
\end{itemize}

\section{Reading files}
\begin{itemize}
    \item \textit{cat}: allow to read a file's content, common options:
        \begin{itemize}
            \item -n line numbering in the output
        \end{itemize}
    \item \textit{more}: display one screen at a time when the file is large
    \item \textit{less}: a version more sophisticated of \textit{more} with
        navigation forward and backward option +F monitor changes in the file
        (monitoring log files)
    \item \textit{head}: allow to print top N number of lines from a file
    \item \textit{tail}: allow to print last N number of lines from a file
\end{itemize}

\section{Files management}
\begin{itemize}
    \item \textit{touch}: create a file
    \item \textit{rm}: delete files or directory (no going back!) 
    \item \textit{mkdir}: create directory
    \item \textit{rmdir}: delete directory if empty
    \item \textit{mv}: move file
    \item \textit{cp}: copy file
    \item \textit{ln}: create symlink with the -s option    
\end{itemize}

\section{Wildcards}
\begin{itemize}
    \item using the Wildcards for file management
    \item * replace any chain of character
    \item ? replace only one character
    \item a-z any character from a to z small caps
    \item 0-9 any digit
\end{itemize}

\section{Finding files}
\begin{itemize}
    \item \textit{which}: search in the PATH to find a file
    \item \textit{locate}: find a path of a binary
    \item \textit{find}: find allow to walk a hierarchy to search for files and
        directories. The main options are:
        \begin{itemize}
            \item \textit{-name}: search by name
            \item \textit{-iname}: search by name case insensitive
            \item \textit{-type f/d/l/s}: search by type (files, directories,
                links, sockets)
            \item \textit{-size}: search by file size
            \item \textit{-user}: search by user
        \end{itemize}
\end{itemize}

\section{Pipeline}
Every program running on the command line in Linux has 3 data streams:
\begin{itemize}
    \item \textit{STDIN(0)}: standard input
    \item \textit{STDOUT(1)}: standard output
    \item \textit{STDERR(2)}: standard error
\end{itemize}
Pipping and redirection means that we may connect these streams between
different programs (STDOUT of one program to the STDIN of another) 

\section{grep, sed, cut and awk}
\begin{itemize}
    \item \textit{grep}: searches text files for a given regex. Common option -r
        for recursive
    \item \textit{sed}: stream editor, text editing on stream text
    \item \textit{cut}: extract a section of text from a line. Common options:
        \begin{itemize}
            \item \textit{-f}: field number
            \item \textit{-d}: field delimiter
        \end{itemize}
    \item \textit{awk}: programming language for text processing. Common
        options:
        \begin{itemize}
            \item \textit{-F}: field separator
            \item \textit{print}: subcommand which outputs the result
        \end{itemize}
\end{itemize}

\section{Comparing files}
\begin{itemize}
    \item \textit{comm}: compares two text files. Output three columns:
        \begin{itemize}
            \item \textit{column 1}: lines that are unique to the first file
            \item \textit{column 2}: lines that are unique to the second file
            \item \textit{column 3}: lines that are shared by both file
        \end{itemize}
        \textit {option -n} where n is either 1, 2 or 3 to suppress the columns
        not needed
    \item \textit{diff}: detect differences between files. Popular format
        options:
        \begin{itemize}
            \item \textit{-c}: context format
            \item \textit{-u}: unified format
            \item \textit{-B}: skip blank line
            \item \textit{-d}: smallest set of differences
        \end{itemize}

\end{itemize}
\section{Editing files from command line}
\begin{itemize}
    \item \textit{nano}: simple text editor not installed by default on linux
        \begin{itemize}
            \item \textit{ctrl + O}: save a file
            \item \textit{ctrl + X}: exit the editor
            \item \textit{ctrl + W}: search for a string
        \end{itemize}
    \item \textit{vi or vim}: powerful text editor installed by default on most
        linux system
        \begin{itemize}
            \item \textit{:q}: quit the file without saving (normal mode)
            \item \textit{:w}: write the file (normal mode)
            \item \textit{i}: entering in insert mode
            \item \textit{ESC}: come back in normal mode
        \end{itemize}
\end{itemize}

\end{document}
